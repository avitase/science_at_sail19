\begin{frame}{Nis Meinert}
    \begin{columns}[T]
        \begin{column}{.6\textwidth}
            \begin{itemize}
                \item LHCb Mitglied seit 2013
                \item 2016: Master of Science (Uni. Rostock)
                \begin{itemize}
                    \item Massenspektroskopie von schweren Baryonen
                \end{itemize}
                \item Seit 2016: Doktorand am Institut für Physik (Uni. Rostock)
                \item Mehrere kürzere Aufenthalte am CERN
                \item Forschungsschwerpunkt: 
                \begin{itemize}
                    \item CP-Verletzung in Zerfällen von schweren Baryonen
                    \item Upgrade des LHCb Detektors
                \end{itemize}
            \end{itemize}
        \end{column}
        \begin{column}{.4\textwidth}
            \centering
            \begin{overpic}[height=.8\textheight,trim=200 0 0 0,clip]{img/me.png}
            \end{overpic}
        \end{column}
    \end{columns}
\end{frame}

\begin{frame}{Des Large Hadron Collider (LHC)}
    \begin{columns}[T]
        \begin{column}{.5\textwidth}
            \textbf{Der Large Hadron Collider (LHC)}
            \begin{itemize}
                \item Wikipedia: \enquote{[...] the most complex experimental facility ever built and the largest single machine in the world.}
                \item Synchrotron (in einem 27\,km langem unterirdischem Ringtunnel): \textbf{beschleunigt} u.a. Protonen auf fast Lichtgeschwindigkeit und \textbf{kollidiert} diese an bestimmten Punkten (zum Beispiel beim LHCb Detektor)
            \end{itemize}
        \end{column}
        \begin{column}{.5\textwidth}
            \centering
            \begin{overpic}[width=\textwidth]{img/cern_map.png}
            \end{overpic}
        \end{column}
    \end{columns}
\end{frame}


